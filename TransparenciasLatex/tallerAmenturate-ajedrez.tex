\documentclass[a4paper,t,xcolor=pst,dvips,colortheme]{beamer}

\input{slidesHeader.tex}

\title[Computadores y Juegos]{¿Cómo juega un computador al ajedrez y por qué no puedo ganarle?}

\author[Pablo Sánchez]{\alert{Pablo Sánchez}}

\institute[I2E]{
		   Dpto. Ingenier{\'i}a Inform{\'a}tica y Electr{\'o}nica \\
		   Universidad de Cantabria \\
		   Santander (Cantabria, España) \\
		   p.sanchez@unican.es
}

\date{}

\begin{document}

\begin{frame}[c]
	\titlepage
	\begin{columns}
		\column{0.50\linewidth}
			\centering
    		\includegraphics[width=.28\textwidth,keepaspectratio=true]{images/istr.eps}
		\column{0.50\linewidth}
			\centering
			\includegraphics[width=.25\textwidth,keepaspectratio=true]{images/uc.eps}
	\end{columns}
\end{frame}

\section{Introducción}

\subsection{Objetivos}

\subsection{Preliminares}

%% Cómo funciona un computador
%% Concepto de algoritmo

\section{Juegos Básicos: Algoritmo Minimax}

\section{Backtracking y Ramificación y Poda}

\section{Redes Neuronales: Jugando Parchís}

\end{document} 